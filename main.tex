\documentclass[12pt,letterpaper]{article}

\title{Git for Economists}
\author{Kristy \& Frank}

\begin{document}
\maketitle
One thing worth stressing is that there is a lot of helpful information about git out there.
We are not aiming to be the most exhaustive or the most authoritative: quite the contrary!
The purpose of this document is twofold: first to \emph{convince} economists that they should be using git and second to make it as easy as possible for them to get started doing \emph{basic stuff}.
\section*{To Do}
\begin{enumerate}
	\item Illustrate some limitations of Dropbox and people's usual workflow
		\begin{itemize}
			\item Conflicted version \emph{with yourself}
			\item Conflicted version with a co-author
			\item Co-author deletes or changes something and it's tricky to undo
			\item ``Packrat'' service now only goes back one year in time
			\item Out of control directories with multiple date-stamped versions, futile attempts to use text files to keep records of what you've done.
		\end{itemize}
	\item Working alone with git
		\begin{itemize}
			\item Getting set up
			\item Git as the ultimate undo button: recovering deleted files, undoing changes selectively
			\item Git as an ``intentional backup'' pushing and pulling with Github
			\item Git as a trail of breadcrumbs: git diff and commit messages to stay organized, remember what you've done, and avoid making duplicates of your files
			\item Trying out new ideas safely with branch-and-merge 
			\item Merge to bring things in from a branch, or fix a conflict with yourself
		\end{itemize}
	\item Collaborating with git and Github
		\begin{enumerate}
			\item Basic push, pull, merge, correct conflicts workflow
			\item Working in separate branches to avoid stepping on each other's toes
			\item Pull requests, etc 
			\item Re-basing
		\end{enumerate}
	\item Miscellaneous
		\begin{itemize}
			\item How to break up a \LaTeX document into pieces to make life easier for yourself, your co-authors, and git
			\item Github pages for your academic website and teaching
		\end{itemize}
\end{enumerate}
This \textit{was} different from what Kristy wrote, but it's better.	
\end{document}
